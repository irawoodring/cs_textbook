\newglossaryentry{computer}{
	name={computer},
	description={A machine.}
}

\newglossaryentry{instruction set}{
	name=instruction set,
	description={The built-in commands that a computer processor can perform.}
}
 
\newglossaryentry{register}{
	name=register,
	description={A small and usually very fast memory location on a computer processor.}
} 

\newglossaryentry{abstraction}{
	name=abstraction,
	description={Creating a new concept (in computing this is usually a new data-type or process-type) by combining lower-level concepts and then providing the combination with a new name.}
}
 
\newglossaryentry{decimal}
{
	name=decimal,
	description={The method of representing numbers using a base of 10.  For the majority of the world this is the usual counting system.}
}

\newglossaryentry{base-10}
{
	name=base-10,
	description={See decimal.}
}

\newglossaryentry{hexadecimal}
{
	name=hexadecimal (or hex),
	description={A method of representing numbers using a base of 16 instead of a base of 10.  Uses digits 0-9 and letters A-F (lower or uppercase).  Often prepended with "0x" to remove any confusion.  For instance, 0x0A (which represents the decimal number '10').}
}

\newglossaryentry{octal}
{
	name=octal,
	description={A base-8 number system.  Makes use of the digits 0-7.  Sometimes written with a lowercase "o" prepended to avoid confusion with decimal, i.e. o42 (which equals 34 in base-10).}
}

\newglossaryentry{virtual machine}{
	name=virtual machine,
	description={A program that runs on a computer and emulates a computer.} 
}

\newglossaryentry{Turing Machine}{
	name=Turing Machine,
	description={The canonical definition of Turing Machine in Computer Science is mathematical model that defines a computing machine.  In this text we choose not to go into the more mathematical concepts from automata and instead refer to the machine described by Turing in his writing, consisting of a tape of infinite length, a read/write head, and a set of instructions.}
}

\newglossaryentry{algorithm}{
	name=algorithm,
	description={
		A set of instructions for solving a probem.
	}
}
\newglossaryentry{parameters}{
	name=parameter,
	description={
		Information given to a function or method so that it may perform its job.  Note that this definition does not exclude out-mode parameters, since they provide information for where a function may store or return data.
	}
}

\newglossaryentry{function}{
	name=function,
	description={A bit of code that together forms an abstraction that solves a particular problem.  Some definitions will include that functions must have all needed data passed to them (i.e. they don't already have access to any needed data).
	}
}

\newglossaryentry{method}{
	name=method,
	description={A bit of code that together forms an abstraction that solves a particular problem.  Some definitions will include that methods are specifically part of object-oriented programming and may have data passed to them, but may also already have access to needed data by way of the object.
	}
}
